\chapter{Clustering à l'aide de l'algorithme Density-based spatial clustering of applications with noise (DBSCAN)}
\section{Introduction}
	Quand nous somme face à un gros ensemble de données dont nous connaissons les caractéristiques mais pas les relations qui lient chaque individu d'un échantillon, nous voudrions être capable de regrouper les instances (point,individu ...) qui sont similaires (petites distances entre eux) dans un sous ensemble bien défini, ce processus est plus communément appelé \textbf{Clustering}.
	\par De manière informelle, le clustering est l'opération de regroupement d'objets dans un groupe compacte nommé \textbf{Cluster} de telle sorte que les membres d'un même cluster soient similaire à un certain point, et arbitrairement différents (grandes distances) des objets d'un autre cluster.
	\par 
	Il existe une multitude d'algorithme de clustering, nous avons décidé d'implémenter l'algorithme Density-based spatial clustering of applications with noise (DBSCAN) pour analyser son comportement, et ainsi critiquer ses forces, ses faiblesses et ainsi pouvoir l'utiliser dans des domaines qui lui conviendraient.
\section{Définitions}
	\paragraph{}
	Comme pours les chapitres précédents, nous devons passer par une section de définitions
	\subsection{E-voisinage}
	\paragraph{}
	\subsection{Core-point}
	\paragraph{}
	\subsection{Point de bord (Border-point)}
	\paragraph{}
	\subsection{Bruit}
	\paragraph{}
	\subsection{Cluster}
	\paragraph{}

\section{Algorithme}
	\paragraph{}
	\section{Implémentation}
	\subsection{Langage de programmation}
	\subsubsection{Schémas d'exécution}
	\subsubsection{Structures de données}
\section{Interface graphique}

\section{Résultats expérimentaux}
	\subsubsection{Choix du dataset}
	\subsubsection{Variations des paramètres}
	\subsubsection{Résultats}
	\subsubsection{Commentaires}

\section{Conclusion}
