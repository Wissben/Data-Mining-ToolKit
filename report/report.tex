% !TeX program = lualatex

\documentclass[12pt]{report}
\usepackage[table,xcdraw]{xcolor}
\usepackage[Glenn]{fncychap}
\usepackage[T1]{fontenc}
\usepackage[francais]{babel}
\usepackage{fontspec}
\usepackage{wrapfig}
\usepackage{graphicx}
\usepackage[a4paper, width=175mm, top=25mm, bottom=25mm]{geometry}
\usepackage{parskip}
\usepackage{enumitem}
\usepackage{listings}
\usepackage{float}
\usepackage[final]{pdfpages}
\usepackage{tocbibind}
\usepackage{tocloft}
\usepackage{xpatch}
\usepackage{amsmath}
\usepackage{amsthm}
\usepackage{amsfonts}
\usepackage{graphics}
\usepackage{framed}
\usepackage{multirow}
\usepackage{graphicx}
\usepackage[utf8x]{inputenc}
\setcounter{secnumdepth}{3} 
\usepackage{mathtools}
\usepackage{amsmath}
\usepackage{tabularx}
\usepackage[ruled,french,onelanguage]{algorithm2e}
\usepackage{tikz}
\usepackage{multirow}
\usepackage[noend]{algpseudocode}
\usepackage[table,xcdraw]{xcolor}

\usepackage{tabularx}  % for tabularx
% to make cells in table have decent spacing
\usepackage{diagbox}
\usetikzlibrary{arrows, positioning, automata}

\usepackage{array}
\newcolumntype{L}[1]{>{\raggedright\let\newline\\\arraybackslash\hspace{0pt}}m{#1}}
\newcolumntype{C}[1]{>{\centering\let\newline\\\arraybackslash\hspace{0pt}}m{#1}}
\newcolumntype{R}[1]{>{\raggedleft\let\newline\\\arraybackslash\hspace{0pt}}m{#1}}


\definecolor{mygreen}{rgb}{0,0.6,0}
\definecolor{mygray}{rgb}{0.5,0.5,0.5}
\definecolor{mymauve}{rgb}{0.58,0,0.82}

\lstset{ %
	backgroundcolor=\color{white},   % choose the background color
	basicstyle=\footnotesize,        % size of fonts used for the code
	breaklines=true,                 % automatic line breaking only at whitespace
	captionpos=b,                    % sets the caption-position to bottom
	commentstyle=\color{mygreen},    % comment style
	escapeinside={\%*}{*)},          % if you want to add LaTeX within your code
	keywordstyle=\color{blue},       % keyword style
	stringstyle=\color{mymauve},     % string literal style
}


\begin{document}
\includepdf[pages=1]{Page_garde.pdf} 
\tableofcontents

\pagenumbering{arabic}
\newpage


\chapter{Introduction et problématique}



\setlist[itemize]{label=\textbullet}
\chapter{Extraction de motifs fréquents à l'aide de l'algorithme Apriori}
	\section{Introduction}
		\paragraph{}
		Souvent confronté à un ensemble de données qui n'ont vraisemblablement pas une régularité ou des sous structures qui se répètent suivant un certain motif, Une des tâches la plus répandue dans le domaine du Data-Mining est l'extraction de ces dits \textbf{Motifs fréquents}.
		\par 
		De façon informelle, un motifs fréquent peut être un item(objet, article ...) une sous-séquences d'items, une sous-structure(sous-graphe, sous-ensemble ...) qui se répète un certain nombre minimum de fois dans la base de données, ce qui lui vaut le nom de motifs \textbf{fréquent}\cite{Han}.
		\par 
		Dans ce qui suit nous allons voir deux algorithmes capables tout deux d'extraire de tels motifs, l'algorithme \textbf{Apriori} \cite{Agrawal:1994:FAM:645920.672836} et l'algorithme FP-Growth \cite{Hanfp}
		
	\section{Définitions}
		\paragraph{}
		Avant d'introduire les deux algorithmes, il faut d'abord définir quelques concepts qui sont intrinsèquement reliés au déroulement de ces deux derniers:
		\subsection*{Items}
			\paragraph{}
			Un item $I_i$ est généralement un attribut associé à un dataset(Taille,Poids,Catégorie...), cet item a un domaine de définition $D_{I_i}$.
		
		\subsection*{Transaction}\label{transaction}
			\paragraph{}
			Une transaction $T_i$ est généralement une instance du dataset, elle se présente comme un ensemble d'items aux quels une valeur à été attribué : 
			$T_i = \lbrace t_1,t_2,...,t_n\rbrace$, on lui associe un identifiant unique $id_{D_i}$.
		
		\subsection*{Support}
			\paragraph{}
			Un support $S$ est un indicateur(une mesure) de combien de fois un ensemble d'item $X$ apparaît dans un dataset $T$, il est définie comme le nombre de transactions $t$ qui contiennent l'itemset $X$ : 
		
			\begin{equation*}
				Support(X) = \frac{|t \in T ; X \subseteq t|}{|T|}
			\end{equation*}
		
	\section{Algorithme}
		\paragraph{}
		Apriori est un algorithme proposé par Agrawal et Srikant en 1994 dans \cite{Agrawal:1994:FAM:645920.672836}, son but est l'extraction de motifs fréquents dans une base de données de transactions \ref{transaction}.
		\par Apriori construit les ensembles d'items candidats à partir d'un ensemble d'items singletons en générant à chaque itérations une extension de ces derniers en ajoutant un item à la fois tout en testant la condition de support minimum ainsi que la condition de sous-motifs fréquent \footnote{Si $M$ est un motif fréquent alors $\forall m_i \in M $ $m_i$ est aussi un item fréquent} pour permettre l'élimination plus rapide des itemsets candidats, l'algorithme s'arrête quand aucune extension ne peut être générée, le pseudo code est le suivant :
		
	
		\begin{algorithm}[H]
			\caption{Apriori}
			\SetKwInOut{Input}{Entrée}\SetKwInOut{Output}{Sortie}
			\SetKwFunction{gen}{GenererCandidats}
			\SetKwFunction{contains}{Contient}
			
			\Input{(T : Ensemble des transactions , $Sup_{min}$ : entier )}
			\Output{(L : Ensemble des items fréquents)}
			\textbf{Var :} \\
			$C_k : $ Itemset des candidats de taille $K$
			$L_k : $ Itemset des items les plus fréquents de taille $K$
			
			
			\Begin
			{
				$L_1$ \gets $\lbrace items les plus fréquent \rbrace$ ;\\
				\For{($k \gets $ ; $L_k \neq \emptyset $ ; $k \gets k+1$)}
				{
					$L_{k+1} \gets $ \gen{$L_k$};\\
					\ForEach{transaction $t \in T$ }
					{
						\For{candidat $c \in C_{k+1}$}
						{
							\If{\contains{$t$,$c$}}
							{
								$compteur[c] \gets compteur[c]+1$
							}
						}
						
					}
					$L_{k+1} \gets \lbrace 	c | c \in C_{k+1} \land compteur[c] \geq Sup_{min}\rbrace$ 	
				}
				
			}
			\KwRet{$\bigcup\limits_{m} L_m ; m = 0,k $}
		\end{algorithm}
	\section{Implémentation}
		\subsection{Langage de programmation}
		\subsubsection{Schémas d'exécution}
		\subsubsection{Structures de données}
	\section{Interface graphique}
	
	\section{Résultats expérimentaux}
		\subsubsection{Choix du dataset}
		\subsubsection{Variations des paramètres}
		\subsubsection{Résultats}
		\subsubsection{Commentaires}
		
	\section{Conclusion}
	
\newpage
\chapter{Classification à l'aide de l'algorithme K plus proches voisins (KNN)}
	\section{Introduction}
	\section{Définitions}
		\paragraph{}
		\subsection{Point}
			\paragraph{}
		\subsection{Distance}\label{transaction}
			\paragraph{}
		\subsection{Voisinage}
			\paragraph{}
		\subsection{Classification}
	
	\section{Algorithme}
		\paragraph{}
	\section{Implémentation}
		\subsection{Langage de programmation}
		\subsubsection{Schémas d'exécution}
		\subsubsection{Structures de données}
	\section{Interface graphique}
	
	\section{Résultats expérimentaux}
		\subsubsection{Choix du dataset}
		\subsubsection{Variations des paramètres}
		\subsubsection{Résultats}
	\subsubsection{Commentaires}
	
	\section{Conclusion}

\newpage
\chapter{Clustering à l'aide de l'algorithme Density-based spatial clustering of applications with noise (DBSCAN)}
\section{Introduction}
\section{Définitions}
\paragraph{}
\subsection{Point}
\paragraph{}
\subsection{Distance}\label{transaction}
\paragraph{}
\subsection{Voisinage}
\paragraph{}
\subsection{Core-point}
\paragraph{}
\subsection{Point de bord (Border-point)}
\paragraph{}
\subsection{Bruit}
\paragraph{}
\subsection{Cluster}
\paragraph{}

\section{Algorithme}
\paragraph{}
\section{Implémentation}
\subsection{Langage de programmation}
\subsubsection{Schémas d'exécution}
\subsubsection{Structures de données}
\section{Interface graphique}

\section{Résultats expérimentaux}
\subsubsection{Choix du dataset}
\subsubsection{Variations des paramètres}
\subsubsection{Résultats}
\subsubsection{Commentaires}

\section{Conclusion}

\newpage

\chapter{Conclusion}
	\paragraph{}
	Au terme de ce projet, nous avons donc pu explorer 3 des aspects du data-mining qui sont : l'extraction de motifs fréquents, la classification des données et le clustering. en implémentant à chaque étapes un algorithme rudimentaire du sous domaine en question, arrivé a ce point nous disposons donc de notre propre outillage pour l'exploration et l'exploitation des ensemble de données.
	\par 
	Il reste toute fois à dresser un bilan récapitulatif qui recense dans chaque partie une analyse de l'algorithme utilisé ainsi que des critiques sur ce dernier.
	\section{Bilan récapitulatif}
		\subsection{Partie I}
		\paragraph{}
		À la fin du chapitre II (voir \ref{apriori}), nous avons pu nous initier à une technique basique d'extraction de motifs fréquents sous forme d'items , et cela depuis un dataset. En implémentant l'algorithme Apriori et en le testant sur un ensemble de benchmark, nous avons pu en tirer les conclusions suivantes : 
		\begin{itemize}
			\item \textbf{Point forts :} 
			 \begin{itemize}
				\item Il est très facile à implémenter, omettant les amélioration des structures de données en terme de temps d'accès (ce qui a poussé à utiliser des structures plus développés que de simple vecteur(tableaux)), la facilité de l'implémentation d'Apriori est une facteur non négligeable quand nous sommes amenés à développer une solution rapidement.
				\item Il est très simple a comprendre, de par sa nativité et son approche qu'on peut qualifier de \textbf{directe}, dans le sens où aucune tentative d'optimisation des opérations n'est effectuée. Il suffit généralement de dérouler un petit exemple à la main pour comprendre comment l'intuition de sa conception à été trouvée. 
				\item Mise à part le dataset choisi pour nos tests, nous avons aussi eu l'occasion de le tester sur un large ensemble d'items, sa complexité temporelle quasi polynomiale donnait d'assez bon résultats lors de la mise à l'échelle, c'est aussi dû au choix de paramètre de façon intelligente qui a permis cela (prendre un support minimum relatif à la taille du dataset par exemple et non pas une constante non adaptée à chaque dataset, de même pour la confiance minimum). 
			\end{itemize}
			\item \textbf{Point faibles : }
				\begin{itemize}
					\item  La génération des candidats à chaque itération est une opération très coûteuse en temps, effectuant des opérations de jointures qui, si elles ne sont pas optimisées, peuvent alourdir le processus dans le cas d'un large dataset.
					\item L'extraction des règle d'association impose le calcul de l'ensemble des sous-ensembles de chaque itemset, c'est aussi une opération très coûteuse en temps.
					\item Le calcul du support minimum impose le parcours de la table des items en entier, et cela à chaque itérations de l'algorithme, de plus l'algorithme assume que cette table soit chargée en mémoire de façon permanente, ce qui peu poser problème si sa taille atteint un seuil critique.
				\end{itemize} 
		\end{itemize}
		\paragraph{Partie II}
		\paragraph{Partie III}
		\paragraph{}
%\input{part2/part2.tex}
\bibliographystyle{ieeetr}
\bibliography{ref.bib}

\end{document}}

